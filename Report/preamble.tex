%%%%%%% This document is a modyfied version of preamble_ida15 given, for free use, to the students who participated in the LaTeX course in 2015. %%%%%%%

\documentclass[
a4paper,
10pt,fleqn,
oneside,openright
]{memoir} 	% Openright aabner kapitler paa hoejresider (openany begge)

%%psw%%
\usepackage{titletoc}
%\titlecontents{}% <section-type>
%  [0pt]% <left>
%  {\bfseries}% <above-code>
%  {\chaptername\ \thecontentslabel:\quad}% <numbered-entry-format>
%  {}% <numberless-entry-format>
%  {\hfill\contentspage}% <filler-page-format>
\setlrmarginsandblock{3cm}{3cm}{*}
\setulmarginsandblock{3cm}{4cm}{*}
% space for header material and footer material
\setheadfoot{14mm}{7mm}
% spacings within the header
\setheaderspaces{*}{3mm}{*}

\checkandfixthelayout[nearest] 						% Oversaetter vaerdier til brug for andre pakker

% \setlrmarginsandblock{3.5cm}{2.5cm}{*}		% \setlrmarginsandblock{Indbinding}{Kant}{Ratio}
% \setulmarginsandblock{2.5cm}{3.0cm}{*}		% \setulmarginsandblock{Top}{Bund}{Ratio}

%	¤¤ Afsnitsformatering ¤¤ %
\setlength{\parindent}{0mm}           		% Stoerrelse af indryk
\setlength{\parskip}{2.5mm}          			% Afstand mellem afsnit ved brug af double Enter
\linespread{1,1}							% Linie afstand

\usepackage{longtable}

%%%% PACKAGES %%%%
\usepackage{listings}
\usepackage{nameref}
% ¤¤ Oversaettelse og tegnsaetning ¤¤ %
\usepackage[utf8]{inputenc}					% Input-indkodning af tegnsaet (UTF8)
\usepackage[T1]{fontenc}					% Output-indkodning af tegnsaet (T1)
\usepackage{ragged2e,anyfontsize}			% Justering af elementer
\usepackage{fixltx2e}						% Retter forskellige fejl i LaTeX-kernen
\usepackage{tabularx}
\usepackage{booktabs}
\usepackage{siunitx}
\usepackage{wrapfig}
																			
% ¤¤ Figurer og tabeller (floats) ¤¤ %
%\usepackage{tikz} % Graphical tool

\usepackage{graphicx} 						% Haandtering af eksterne billeder (JPG, PNG, PDF)
\usepackage{multirow}                		% Fletning af raekker og kolonner (\multicolumn og \multirow)
\usepackage{colortbl} 						% Farver i tabeller (fx \columncolor, \rowcolor og \cellcolor)
\usepackage[xcdraw]{xcolor}
%\usepackage[dvipsnames]{xcolor}				% Definer farver med \definecolor. Se mere: %http://en.wikibooks.org/wiki/LaTeX/Colors
\usepackage{flafter}						% Soerger for at floats ikke optraeder i teksten foer deres reference
\let\newfloat\relax 						% Justering mellem float-pakken og memoir
\usepackage{float}							% Muliggoer eksakt placering af floats, f.eks. \begin{figure}[H]
\usepackage{adjustbox}				% Muliggoer left og right placering af billeder
%\usepackage{eso-pic}						% Tilfoej billedekommandoer paa hver side
\usepackage{wrapfig}						% Indsaettelse af urer omsvoebt af tekst. \begin{wrapfigure}{Placering}{Stoerrelse}
%\usepackage{multicol}         	        	% Muliggoer tekst i spalter
%\usepackage{rotating}						% Rotation af tekst med \begin{sideways}...\end{sideways}
%\usepackage{caption} 						% Giver mulighed for at redigere bredden på captions, hvis dit billede er mindre end caption

% ¤¤ Matematik mm. ¤¤
\usepackage{amsmath,amssymb,stmaryrd} 		% Avancerede matematik-udvidelser
\usepackage{mathtools}						% Andre matematik- og tegnudvidelser
\usepackage{textcomp}                 		% Symbol-udvidelser (f.eks. promille-tegn med \textperthousand )
\usepackage{siunitx}						% Flot og konsistent praesentation af tal og enheder med \si{enhed} og \SI{tal}{enhed}
\sisetup{output-decimal-marker = {,}}		% Opsaetning af \SI (DE for komma som decimalseparator) 
%\usepackage[version=3]{mhchem}
\usepackage{fix-cm} % Fix for cm 				% Kemi-pakke til flot og let notation af formler, f.eks. \ce{Fe2O3}
%\usepackage{rsphrase}						% Kemi-pakke til RS-saetninger, f.eks. \rsphrase{R1}

% ¤¤ Referencer og kilder ¤¤ %
\usepackage[danish]{varioref}				% Muliggoer bl.a. krydshenvisninger med sidetal (\vref)
\usepackage[square, numbers, authoryear, super]{natbib}
%\usepackage[backend=bibtex]{biblatex}
\bibliographystyle{Projektrapport/natbib}
%\usepackage[square,numbers]{natbib}							% Udvidelse med naturvidenskabelige citationsmodeller
%\usepackage{xr}							% Referencer til eksternt dokument med \externaldocument{<NAVN>}
%\usepackage{glossaries}					% Terminologi- eller symbolliste (se mere i Daleifs Latex-bog)

% ¤¤ Misc. ¤¤ %
\usepackage{listings}						% Placer kildekode i dokumentet med \begin{lstlisting}...\end{lstlisting}

\usepackage{lipsum}							% Dummy text \lipsum[..]
\usepackage[shortlabels]{enumitem}			% Muliggoer enkelt konfiguration af lister
\usepackage{pdfpages}						% Goer det muligt at inkludere pdf-dokumenter med kommandoen \includepdf[pages={x-y}]{fil.pdf}	

\pdfoptionpdfminorversion=6					% Muliggoer inkludering af pdf dokumenter, af version 1.6 og hoejere
\pretolerance=2500 							% Justering af afstand mellem ord (hoejt tal, mindre orddeling og mere luft mellem ord)

% Kommentarer og rettelser med \fxnote. Med 'final' i stedet for 'draft' udloeser hver note en error i den faerdige rapport.
\usepackage[footnote,draft,danish,silent,nomargin]{fixme}		


%%%% CUSTOM SETTINGS %%%%

% ¤¤ Marginer ¤¤ %

% marginer roder man ikke med manuelt nede i dokumentet /daleif


% ¤¤ TOC ¤¤ %

\setpnumwidth{3em}
\setrmarg{3.55em}



% ¤¤ Litteraturlisten ¤¤ %

% \bibpunct[,]{[}{]}{;}{a}{,}{,} 				% Definerer de 6 parametre ved Harvard henvisning (bl.a. parantestype og seperatortegn)
%\bibliographystyle{bibtex/harvard}			% Udseende af litteraturlisten.

% ¤¤ Indholdsfortegnelse ¤¤ %
\setsecnumdepth{paragraph}		 			% Dybden af nummerede overkrifter (part/chapter/section/subsection)
\maxsecnumdepth{paragraph}					% Dokumentklassens graense for nummereringsdybde
\settocdepth{paragraph} 					% Dybden af indholdsfortegnelsen

% ¤¤ Lister ¤¤ %
\setlist{
  topsep=0pt,								% Vertikal afstand mellem tekst og listen
  itemsep=-1ex,								% Vertikal afstand mellem items
} 

% ¤¤ Visuelle referencer ¤¤ %
\usepackage[colorlinks]{hyperref}			% Danner klikbare referencer (hyperlinks) i dokumentet.
\hypersetup{colorlinks = true,				% Opsaetning af farvede hyperlinks (interne links, citeringer og URL)
    linkcolor = black,
    citecolor = black,
    urlcolor = black
}

% ¤¤ Opsaetning af figur- og tabeltekst ¤¤ %
\captionnamefont{\small\bfseries\itshape}	% Opsaetning af tekstdelen ('Figur' eller 'Tabel')
\captiontitlefont{\small}					% Opsaetning af nummerering
\captiondelim{. }							% Seperator mellem nummerering og figurtekst
\hangcaption								% Venstrejusterer flere-liniers figurtekst under hinanden
\captionwidth{\linewidth}					% Bredden af figurteksten
\setlength{\belowcaptionskip}{0pt}			% Afstand under figurteksten
		
% ¤¤ Opsaetning af listings ¤¤ %
\definecolor{commentGreen}{RGB}{34,139,24}
\definecolor{stringPurple}{RGB}{155,104,2}
\definecolor{ase_blue}{RGB}{10,55,136}
\definecolor{bluekeywords}{rgb}{0,0,1}
\definecolor{types}{rgb}{0.17,0.57,0.68}
\definecolor{redstrings}{rgb}{0.64,0.08,0.08}

\lstset{language=csh,					% Sprog
	captionpos=b,
	showspaces=false,
	showtabs=false,
	breaklines=true,
	showstringspaces=false,
	breakatwhitespace=true,
	escapeinside={(*@}{@*)},
	commentstyle=\color{commentGreen},		% Opsaetning af kommentarer
	morecomment=[l]{//}, 					%use comment-line-style!
	morecomment=[s]{/*}{*/}, 				%for multiline comments
	stringstyle=\color{stringPurple},		% Opsaetning af strenge
	keywordstyle=\color{bluekeywords},
	stringstyle=\color{redstrings},
	basicstyle=\ttfamily\small,
	showstringspaces=false,					% Mellemrum i strenge enten vist eller blanke
	numbers=left, numberstyle=\tiny,		% Linjenumre
	extendedchars=true, 					% Tillader specielle karakterer
	columns=flexible,						% Kolonnejustering
	breaklines, breakatwhitespace=true,		% Bryd lange linjer
	morekeywords={  abstract, event, new, struct,
                as, explicit, null, switch,
                base, extern, object, this,
                bool, false, operator, throw,
                break, finally, out, true,
                byte, fixed, override, try,
                case, float, params, typeof,
                catch, for, private, uint,
                char, foreach, protected, ulong,
                checked, goto, public, unchecked,
                class, if, readonly, unsafe,
                const, implicit, ref, ushort,
                continue, in, return, using,
                decimal, int, sbyte, virtual,
                default, interface, sealed, volatile,
                delegate, internal, short, void,
                do, is, sizeof, while,
                double, lock, stackalloc,
                else, long, static,
                enum, namespace, string},
}

% ¤¤ Kapiteludssende ¤¤ %
\definecolor{numbercolor}{gray}{0.7}		% Definerer en farve til brug til kapiteludseende
\newif\ifchapternonum

\makechapterstyle{jenor}{					% Definerer kapiteludseende frem til ...
  \renewcommand\chapterheadstart{}
  \renewcommand\beforechapskip{0pt}
  \renewcommand\printchaptername{}
  \renewcommand\printchapternum{}
  \renewcommand\printchapternonum{\chapternonumtrue}
  \renewcommand\chaptitlefont{\fontfamily{pbk}\fontseries{db}\fontshape{n}\fontsize{20}{30}\selectfont\raggedright}
  \renewcommand\chapnumfont{\fontfamily{pbk}\fontseries{m}\fontshape{n}\fontsize{0.6in}{0in}\selectfont\color{numbercolor}}
  \renewcommand\printchaptertitle[1]{%
    \noindent
    \ifchapternonum
    \begin{tabularx}{\textwidth}{X}
    {\let\\\newline\chaptitlefont ##1\par} 
    \end{tabularx}
    \par\vskip-2.5mm\hrule
    \else
    \begin{tabularx}{\textwidth}{Xl}
    {\parbox[b]{\linewidth}{\chaptitlefont ##1}} & \raisebox{-15pt}{\chapnumfont \thechapter}
    \end{tabularx}
    \par\vskip2mm\hrule
    \fi
  }
}											% ... her

\chapterstyle{jenor}						% Valg af kapiteludseende - Google 'memoir chapter styles' for alternativer
%\chapterstyle{southall}{%
%	 \renewcommand{\afterchaptertitle}{{%
%	 		\par\offinterlineskip\hbox{\hspace*{-\marginparwidth}\rule{0.7\textwidth}{5pt}}}}
%}
%\renewcommand*{\chapterheadendvskip}{\vspace{10cm}}

%\chapterstyle{tandh}

% Afstand omkring titler
\usepackage[compact]{titlesec}
\titlespacing*{\section}{0pt}{5pt}{0pt}
\titlespacing*{\subsection}{0pt}{5pt}{0pt}
\titlespacing*{\subsubsection}{0pt}{5pt}{0pt}

% ¤¤ Sidehoved ¤¤ %

\makepagestyle{Uni}							% Definerer sidehoved og sidefod udseende frem til ...
\makepsmarks{Uni}{%
	\createmark{chapter}{left}{shownumber}{}{. \ }
	\createmark{section}{right}{shownumber}{}{. \ }
	\createplainmark{toc}{both}{\contentsname}
	\createplainmark{lof}{both}{\listfigurename}
	\createplainmark{lot}{both}{\listtablename}
	\createplainmark{bib}{both}{\bibname}
	\createplainmark{index}{both}{\indexname}
	\createplainmark{glossary}{both}{\glossaryname}
}
\nouppercaseheads											% Ingen Caps oenskes
\lstdefinestyle{makefile}
{
    numberblanklines=false,
    language=make,
    tabsize=4,
    keywordstyle=\color{red},
    identifierstyle= %plain identifiers for make
}

\definecolor{maroon}{rgb}{0.5,0,0}
\definecolor{darkgreen}{rgb}{0,0.5,0}
\lstdefinelanguage{XML}
{
  basicstyle=\ttfamily,
  morestring=[s]{"}{"},
  morecomment=[s]{?}{?},
  morecomment=[s]{!--}{--},
  commentstyle=\color{darkgreen},
  moredelim=[s][\color{black}]{>}{<},
  moredelim=[s][\color{red}]{\ }{=},
  stringstyle=\color{blue},
  identifierstyle=\color{maroon}
}
%Stil for kode der er brugt i dokumentationen:
%\lstdefinestyle{customc}{
%  belowcaptionskip=1\baselineskip,
%  breaklines=true,
%  frame=L,
%  xleftmargin=\parindent,
%  language=C,
%  showstringspaces=false,
%  basicstyle=\footnotesize\ttfamily,
%  keywordstyle=\bfseries\color{green!40!black},
%  commentstyle=\itshape\color{purple!40!black},
%  identifierstyle=\color{blue},
%  stringstyle=\color{orange},
%}
%\lstset{escapechar=@,style=customc}\pagestyle{}

%Noget med at holde footnotes i bunden
\usepackage[bottom]{footmisc}
%Her er en streg
\renewcommand{\footnoterule}{%
	\kern -3pt
	\hrule width \textwidth height 1pt
	\kern 2pt
}

%AU LOGO - Er kommenteret ud. Hvis dette skal med, fjern % for de to øverste linjer

%\makeevenhead{Uni}{\includegraphics[height=1cm]{Dokumentation/Forside/aulogo}}{}{\leftmark}				% Definerer lige siders sidehoved (\makeevenhead{Navn}{Venstre}{Center}{Hoejre})
%\makeoddhead{Uni}{\rightmark}{}{\includegraphics[height=1cm]{Dokumentation/Forside/aulogo}}			% Definerer ulige siders sidehoved (\makeoddhead{Navn}{Venstre}{Center}{Hoejre})
\makeevenfoot{Uni}{\thepage}{}{}							% Definerer lige siders sidefod (\makeevenfoot{Navn}{Venstre}{Center}{Hoejre})
\makeoddfoot{Uni}{}{}{\thepage}								% Definerer ulige siders sidefod (\makeoddfoot{Navn}{Venstre}{Center}{Hoejre})
\makeheadrule{Uni}{\textwidth}{0.5pt}						% Tilfoejer en streg under sidehovedets indhold
%\makefootrule{Uni}{\textwidth}{0.5pt}{1mm}					% Tilfoejer en streg under sidefodens indhold
%\setlength{\headheight}{40pt}

\copypagestyle{Unichap}{Uni}								% Sidehoved for kapitelsider defineres som standardsider, men med blank sidehoved
\makeoddhead{Unichap}{}{}{}
\makeevenhead{Unichap}{}{}{}
\makeheadrule{Unichap}{\textwidth}{0pt}

\makeevenfoot{Uni}{\thepage}{}{}
\makeoddfoot{Uni}{}{}{\thepage}

\aliaspagestyle{chapter}{Unichap}							% Den ny style vaelges til at gaelde for chapters
															% ... her
															
\pagestyle{Uni}												% Valg af sidehoved og sidefod (benyt "plain" for ingen sidehoved/fod

% ¤¤ Billede hack ¤¤ %										% Indsaet figurer nemt med \figur{Stoerrelse}{Fil}{Figurtekst}{Label}
\newcommand{\figur}[4]{
		\begin{figure}[H] \centering
			\includegraphics[width=#1\textwidth]{billeder/#2}
			\caption{#3}\label{#4}
		\end{figure} 
}

% ¤¤ Specielle tegn ¤¤ %
\newcommand{\decC}{^{\circ}\text{C}}
\newcommand{\dec}{^{\circ}}
\newcommand{\m}{\cdot}


%%%% ORDDELING %%%%

\hyphenation{}


% gør det nemmere at se overfull fejl
\overfullrule=2cm