\pagestyle{Uni}

\chapter{Requirements specification}

\section{Vision}
	The vision with the project is to make a social game that is played as co-op so that a fellow-felling is created.  The game is intended for physical attendance where each player has their own android device. However, it is also possible to play online with friends, since all communication is intended to be happening over the Internet. In the following sections there are various Personans and scenarios that will try to give an impression on how a product like the one to be developed could be used.


\section{Personas}
\textbf{Christian} is 25 years old and lives in the hearth of Aarhus, he is dreaming of becoming a teacher. He loves to sit together with his friends late into the evening and play board games at a friends place, as the location is better and there is more room. Christian is quite technology interested and he always has the latest Samsung phone. In addition, he loves to spend his spare time playing computer games. The way he gets around in Aarhus is by using his bike. Christian owns the game that the group of friends all think is the most fun to play but he is tired of having to transport the game back and forth as it rather unhandy.

\textbf{Lærke} is 23 years old and lives in Manchester, she is currently an exchange students during her education in state economics. Lærke is very socially established and has a lot of good friends in Denmark. She keeps in touch with her friends in Denmark by using her android tablet to play different games.

\section{Scenarios}
\textbf{Christian}:
It is Thursday evening and Christian has taken his board game under his arm, like he does every Thursday and mounted his bike. Before he hits Thomas's house, he bikes towards Netto to get ready for a long evening. However, he thinks it's annoying to drag the game through Netto and in addition, it also limits the amount of supplies he can carry on his bike. When Christian arrives at Thomas and opens the game box, he can see how all the cards are messed up after the bike ride, after a while they have finally got all the content of the box sorted and can start playing. After playing for a while, a friend makes an unauthorized move, which is first discovered later. This move annoys Christian, as this will have an impact on the outcome of the game. Late in the evening they will have to stop the game as they go to school the next day and have to write down how far they have come.

\textbf{Lærke}:
It is Friday evening and Lærke can not afford to go out to town. Instead, she opens her tablet and finds the "Play store" to find a new co-op game she can play with her Danish friends. She finds the game "BackFight". Lærke writes to her Danish friends if they want to play with her. They quickly download the game and start a game. Time flies away and they are doing their second game, unfortunately, they have to stop when one of them need to wake up early the next day. The following day, they all want to continue the game. They open the app and choose their game from the day before and can continue from where they stopped.

\section{User interface design}
This section will show the ideas about how the application was supposed to look and feel when the design was made in the very start of the project.

\begin{figure}[ht!]
	\centering
	\includegraphics[width=90mm]{images/Gui.png}
	\caption{Menu navigation \label{navigation}}
\end{figure}

\begin{figure}[ht!]
	\centering
	\includegraphics[width=90mm]{images/GameView.png}
	\caption{Game layout \label{layout}}
\end{figure}

